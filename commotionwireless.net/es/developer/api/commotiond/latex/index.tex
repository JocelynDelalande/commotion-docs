This embedded library and daemon are the start of what will form the new core of the Commotion project on embedded platforms. {\itshape This is extremely pre-\/alpha software that does not do anything yet, but which is in rapid development.}

\section*{Build it }

N\+O\+T\+E\+: this project uses C\+Make, so you must have that installed on your system, as well as subversion and sqlite3 header files (libsqlite3-\/dev).


\begin{DoxyEnumerate}
\item Clone the repository.
\item cd commmotiond
\item mkdir build
\item cd build
\item cmake ..
\item make
\item sudo make install
\end{DoxyEnumerate}

\section*{Commotion components }

This repository includes 3 main components\+:

\subsection*{libcommotion }

libcommotion is a high level library that contains a C A\+P\+I with all of the tools necessary to create a variety of mesh networks and mesh networking applications, without needing to deal with all of the specifics of configuring each and every type of addressing scheme and mesh networking daemon available. Bindings for other languages like Java or Python are forthcoming.

\subsection*{commotiond }

commotiond is an implementation of libcommotion, in the form of a superserver daemon that creates and manages parent processes for a variety of mesh-\/networking related daemons and services based on a common configuration store. It (will) support a variety of types of plugins and extensions, including\+:


\begin{DoxyItemize}
\item operating-\/system specific extensions for interfacing with the wireless subsystem
\item schemas for mesh network detection
\item A\+P\+Is for different kinds of messaging infrastructures (dbus, ubus, J\+N\+I)
\item drivers for different kinds of mesh networking daemons (olsrd, babeld, servald)
\end{DoxyItemize}

\subsection*{commotion }

The commotion application is a simple command-\/shell interface for managing the commotiond daemon. 